\chapter*{謝辞}
% \label{ch:ack}
まず、本研究を行うにあたって、名古屋大学宇宙地球環境研究所の様々な先生方にご指導とご尽力を賜りました。
水野 亮 教授には、データ解析について基礎から教えていただき、精神的につらい時期には研究以外でもたくさんのサポートをしてくださいました。
長濱 智生 准教授には授業や持田研究室との共同セミナー、毎週の全体ミーティングなどで地球大気に関する基礎やデータ解析のアドバイスをいただきました。
中島 拓 助教には、プレゼン準備の打ち合わせや学会の予稿の添削など、研究発表の仕方について指導していただきました。
塩川 和夫 教授には、電磁気圏セミナーなどで研究の助言をいただき、精神的につらい時期には学内の保健管理室にとりあっていただくなどのサポートをいただきました。
三好 由純 教授には、POES衛星の電子フラックスデータの扱い方や、OMNI Data Setの使い方で様々な助言をいただきました。
また、電磁気圏研究部や持田研究室の方には、セミナーでたくさんの助言をいただきました。
\par

研究室生活については、秘書の原田 真乃 さんに快適な研究室環境を作っていただきました。
また、水野研究室および塩川研究室の先輩・同期・後輩の皆さんには、研究の相談だけでなく、パーティやゲームなどの楽しい時間を研究室で過ごすことができました。\par

名古屋大学総合保健体育科学センター保健科学部の古橋 忠晃 准教授には、医学的なご専門の立場から、研究を行う上での体調管理について様々なサポートをいただきました。
また、学部生として名古屋大学に入学したときからの同期である、小松 大祐 さんには、心が折れそうなときに何度も激励をいただきました。\par

最後になりますが、ここまで紆余曲折ありながら、勉学と研究に集中できるように支えていだだいた私の家族に心より感謝いたします。
