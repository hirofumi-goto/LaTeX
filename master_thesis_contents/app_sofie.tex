\chapter{SOFIE}
\label{app:sofie}
SOFIE(Solar Occultation for Ice Experiment)は2007年3月から2023年3月まで、NASAのAIM(Aeronomy of Ice in the Mesosphere)衛星に搭載されて運用された~\cite{russell2009aeronomy,sofie2006sdl}。
AIM衛星は、極中間圏雲(PMC: Polar Mesospheric Cloud)の研究のために打ち上げられ、太陽同期軌道で周回する。
SOFIEでは、とくにPMCが形成される領域の大気における分子や温度、微粒子を観測するために設計され、5種類の分子(\ce{H2O}・\ce{CO2}・\ce{O3}・\ce{CH4}・\ce{NO})とPMCの消失を11の波長で観測し、流星により発生した粒子について3つの波長で観測している。
これらの観測からそれぞれの高度プロファイルデータが取得され、高度分解能はおよそ$2\ \mathrm{km}$である。
SOFIEは太陽掩蔽法を用いて、主に極域において観測が行われており、SOFIEに対して太陽が日の出や日の入りをする際に、地球大気の縁をかすめるように通過した太陽光について赤外分光観測を行う。
本研究においては、Version 1.3 Dataを用いており、\url{http://sofie.gats-inc.com/swdocs}より取得した。
トロムソのミリ波分子計による\ce{NO}の柱密度の導出を行った期間においては、トロムソ付近の緯度(およそ65 - 80\textdegree Nの範囲)で日の入りの観測を行っていたため、この観測データを用い、\ce{NO}の高度プロファイルデータを比較対象として用いた。
