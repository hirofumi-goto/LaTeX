\chapter{まとめ}
% \label{ch:conclusion}
我々のグループは、太陽活動に伴って極域に降り込む高エネルギー粒子が窒素酸化物($\mathrm{NO}_x$)やオゾンなどの中層大気中の微量分子に及ぼす影響を調べるため、極域でミリ波大気観測を行っている。
先行研究(Isono et al.2014)では、ミリ波分光を用いた昭和基地での一酸化窒素(\ce{NO})のモニタリング観測において、季節変化に伴う長期変動の他、冬期に高エネルギー粒子の影響によると考えられる短期変動が確認された。
しかし、夏期は高エネルギー粒子の降り込みだけでなく、太陽光での光解離による影響もあり、これらを切り分けることが難しかった。
そこで我々のグループでは、季節が逆転する北半球の極域にも着目した。
北極域での観測も行うことで、光解離の影響を受けない夜間の観測時間を増やすと同時に、両極域からの比較観測を行うことを構想した。
本研究では、両極域での相補的な観測を実現するために、新設・改良されたミリ波分光計で実際に取得されたデータを精査して統一的な解析手法の開発を行った。
さらにこの解析によりNOの柱密度の導出を行い、種々の地上・衛星観測による電離圏・磁気圏の観測データと比較することにより、その時地球に降り込んできた高エネルギー粒子の物理量および磁気圏の物理状態と\ce{NO}増加量の関係を明らかにした。\par

具体的には、ミリ波分光計の仕様が異なるノルウェー・トロムソと南極・昭和基地両方のスペクトルデータの解析手法を検討した。
トロムソの分光計では\ce{NO}のスペクトルを十分な精度で得るためには積分時間は24時間が必要であったが、昭和基地に設置された新たな分光計では、\ce{NO}の6本の超微細構造線から導出される柱密度を平均することで、積分時間は12時間に短縮できることが分かった。
さらに昭和基地での柱密度の誤差の平均はトロムソでの観測と比べて20\% 小さくなり、時間分解能を高くしながら、さらに柱密度の誤差を小さくすることに成功した。\par

このようにして得られたミリ波分光計による\ce{NO}の観測データについて、トロムソは2018年12月26日から2019年3月10日までの75日間、昭和基地は2023年3月22日から31日までの10日間を解析し、得られた\ce{NO}の柱密度変動について考察した。
まずトロムソで観測された\ce{NO}の柱密度変動について、SOFIE衛星による観測での\ce{NO}の高度プロファイルデータから導出した柱密度と比較を行った結果、\ce{NO}の柱密度の時間変動については、その傾向がよく一致していることが確認できた。
しかし、SOFIEの観測から導出された\ce{NO}の柱密度の値は、ミリ波分光計から導出した\ce{NO}の柱密度と比べると、およそ1.65倍大きなものとなっていた。
これの原因の1つとしては、ミリ波観測データの解析で\ce{NO}の柱密度を導出する際に仮定している大気温度の値が妥当でない可能性があった。
そこで仮定する大気温度の再検討をしたが、これだけで、SOFIEとの値の差を説明することはできなかった。\ce{NO}の柱密度を導出する際に、大気温度を高度ごとに設定することで、このオフセットは改善できる可能性がある。\par

次に、両極で得られたミリ波分光計による\ce{NO}の柱密度の時間変動の結果を地磁気擾乱の目安となるDst指数と比較した。
その結果、トロムソにおいて\ce{NO}の柱密度の急激な増加がみられた2019年2月1日~2019年2月4日の期間と、昭和基地において\ce{NO}の柱密度の急激な増加がみられた2023年3月23日〜2023年3月24日の期間に対応してDst指数の負方向への変化がみられたため、観測された\ce{NO}の増加は、磁場の擾乱により加速された電子の影響によると考えた。\par

そこで、その期間に実際に電子の降り込みがあったか確認するために、POES衛星による観測での電子フラックスデータと比較を行った。
POESの電子フラックスデータのうちトロムソ・昭和基地付近に降り込むと考えられるデータのみを用いた。
その結果、トロムソ・昭和基地におけるすべての\ce{NO}の柱密度の増加に共通して、比較的エネルギーの大きい$>287\ \mathrm{keV}$における電子の増加が確認できた。
一方、比較的エネルギーの小さい$>40\ \mathrm{keV}$や$>130\ \mathrm{keV}$の電子フラックスのみが増加している期間では、顕著な\ce{NO}の増加がみられなかった。\par

最後に、降り込む粒子の物理量および磁気圏の物理状態と\ce{NO}増加量の関係を明らかにするため、磁気圏と電離圏の様子を調べることができるOMNI Data Setを用いた比較を行った。
ミリ波の観測ではトロムソ・昭和基地で\ce{NO}の柱密度の増加を合計4回確認することができたが、その内の2回については、磁気嵐の主相である時期、1回は磁気嵐の回復相かつ高速太陽風の到達した時期であることが分かった。
またトロムソにおいては、\ce{NO}の柱密度の増加が確認された2つの時期どちらにおいても、地球磁場の南北成分のゆらぎがあり、サブストームが活発で高速太陽風が吹いていることが確認できた。
昭和基地においては、磁気嵐の発生と対応して\ce{NO}の柱密度が増加した2023年3月23日〜2023年3月24日において地球磁場が南方向を向いており、プロトンの密度も上昇していることが確認できたが、高速太陽風は確認できなかった。
もう1つ柱密度の増加が確認できた2023年3月25日においては、SYM/H(もしくはDst指数)の値をみると磁気嵐は回復相にあたるが、高速太陽風があることが確認できた。
これより、磁気圏から回復する期間であっても高速太陽風が到達していると電子の降り込みがあり、\ce{NO}の増加に影響を与えると考えられる。
