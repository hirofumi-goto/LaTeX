\chapter{まとめ}
% \label{ch:conclusion}
我々のグループでは、南極域だけでなく季節が逆転する北半球の極域にも着目し、両極域での比較観測を実現するために、実際に取得されたデータを精査して統一的な解析手法の開発を行った。
さらに、その時地球に降り込んできた高エネルギー粒子がどのように\ce{NO}分子の時間変動に影響を与えているかを調べた。
\par
本研究では、ミリ波分光計の仕様が異なる昭和基地とトロムソ両方のスペクトルデータの解析手法を検討した後、トロムソについては2018年12月26日から2019年3月10日までの75日間、昭和基地については2023年3月22日から31日までの10日間にわたる\ce{NO}の観測データの中から\ce{NO}の柱密度の導出を行って、その変動について考察した。
\par
トロムソの分光計では積分時間は24時間であったが、\ce{NO}の6本の超微細構造線から導出される柱密度を平均することで、昭和基地における積分時間は12時間と短くした。
昭和基地での柱密度の誤差の平均はトロムソでの観測と比べて20\% 小さくなり、時間分解能を小さくしながら柱密度の誤差を小さくすることができた。
解析の期間には磁場の擾乱により加速された電子の影響とみられる\ce{NO}の増加が確認できた。
\par
トロムソで観測されたミリ波分光による\ce{NO}の柱密度について、SOFIEによる衛星観測での\ce{NO}の密度に関する高度プロファイルデータから導出した柱密度と比較を行った結果、\ce{NO}の柱密度の時間変動について傾向が一致していることが分かった。
しかし、SOFIEから導出した\ce{NO}の柱密度は、ミリ波分光計から導出した\ce{NO}の柱密度と比べておよそ1.65倍されたものとなっていた。
これの原因の一部として、\ce{NO}の柱密度を導出する際に仮定した大気温度の値が妥当でない可能性があった。
ミリ波分光計を用いた\ce{NO}の柱密度の導出において仮定する大気温度の再検討をしたが、これだけで、SOFIEから導出した\ce{NO}の柱密度との値の差を説明することはできなかった。\ce{NO}の柱密度を導出する際には、大気温度を高度ごとに設定することも考慮に入れる必要がある。
また、POESによる衛星観測での電子フラックスデータの中でも、トロムソ・昭和基地付近に降り込むと考えられるデータを用いて、ミリ波分光による\ce{NO}の柱密度の時間変動と比較を行った。
電子フラックスの総量だけでなく、比較的エネルギーの大きい$>287\ \mathrm{keV}$における電子の増加量が\ce{NO}の増加の良い指標となることが明らかになった。
比較的エネルギーの小さい$>40\ \mathrm{keV}$, $>130\ \mathrm{keV}$の電子フラックスの増加があった期間では、顕著な\ce{NO}の増加がみられなかった。
\ce{NO}の柱密度の増加に寄与しない電子フラックスの増加の要因を探るため、磁気圏と電離層の様子を調べることができるOMNI Data Setを用いた比較を行った。
トロムソにおいては、\ce{NO}の柱密度の増加が確認された2つの時期どちらにおいても、地球磁場の南北成分のゆらぎがあり、サブストームが活発で高速太陽風が吹いていることが確認できた。
これらの影響により電子が加速され、\ce{NO}の増加につながったと考えられる。
昭和基地においては、磁気嵐の発生と対応して\ce{NO}の柱密度が増加した2023年3月23日〜2023年3月24日において地球磁場が南方向を向いており、プロトンの密度も上昇していることが確認できたが、高速太陽風は確認できなかった。
もう1つ柱密度の増加が確認できた2023年3月25日においては、SYM/H(もしくはDst指数)の値をみると磁気嵐は回復相にあたるが、高速太陽風があることが確認できた。
これは、高速太陽風の影響で電子が加速され\ce{NO}の増加につながった可能性が考えられる。
