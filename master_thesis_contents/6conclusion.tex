\chapter{まとめ}
% \label{ch:conclusion}
我々のグループでは、両極域での相補的な観測を実現するために、北極域において分光計の新設、南極域において分光計の改良を行っている。
そこで、新設・改良されたミリ波分光計のためのデータ解析手法を確立した。
さらに、その解析結果から、その時地球に降り込んできた高エネルギー粒子の物理量および磁気圏の物理状態と\ce{NO}増加量の関係を明らかにした。\par

本研究では、ミリ波分光計の仕様が異なる昭和基地とトロムソ両方のスペクトルデータの解析手法を検討した後、トロムソについては2018年12月26日から2019年3月10日までの75日間、昭和基地については2023年3月22日から31日までの10日間にわたる\ce{NO}の観測データの中から\ce{NO}の柱密度の導出を行って、その変動について考察した。\par

トロムソの分光計では積分時間は24時間であったが、\ce{NO}の6本の超微細構造線から導出される柱密度を平均することで、昭和基地における積分時間は12時間と短くした。
昭和基地での柱密度の誤差の平均はトロムソでの観測と比べて20\% 小さくなり、時間分解能を小さくしながら柱密度の誤差を小さくすることができた。\par

トロムソで観測されたミリ波分光による\ce{NO}の柱密度について、SOFIEによる衛星観測での\ce{NO}の高度プロファイルデータから導出した柱密度と比較を行った結果、\ce{NO}の柱密度の時間変動について傾向が一致していることが分かった。
しかし、SOFIEから導出した\ce{NO}の柱密度は、ミリ波分光計から導出した\ce{NO}の柱密度と比べておよそ1.65倍されたものとなっていた。
これの原因の一部として、\ce{NO}の柱密度を導出する際に仮定した大気温度の値が妥当でない可能性があった。
ミリ波分光計を用いた\ce{NO}の柱密度の導出において仮定する大気温度の再検討をしたが、これだけで、SOFIEから導出した\ce{NO}の柱密度との値の差を説明することはできなかった。\ce{NO}の柱密度を導出する際には、大気温度を高度ごとに設定することも考慮に入れる必要がある。\par

解析期間に地磁気擾乱があったかどうかを調べるため、ミリ波分光による\ce{NO}の柱密度の時間変動をDst指数の時間変動と比較した。
その結果、トロムソにおける\ce{NO}の柱密度の急激な増加がみられた2019年2月1日~2019年2月4日の期間、昭和基地における\ce{NO}の柱密度の急激な増加がみられた2023年3月23日〜2023年3月24日の期間に対応してDst指数の変化がみられ、\ce{NO}の増加が磁場の擾乱により加速された電子の影響と考えた。\par

実際に電子の降り込みがあったか確認するために、POESによる衛星観測での電子フラックスデータの中でも、トロムソ・昭和基地付近に降り込むと考えられるデータを用いて、ミリ波分光による\ce{NO}の柱密度の時間変動と比較を行った。
その結果、トロムソ・昭和基地におけるすべての\ce{NO}の柱密度の増加に共通して、比較的エネルギーの大きい$>287\ \mathrm{keV}$における電子の増加が確認できた。
比較的エネルギーの小さい$>40\ \mathrm{keV}$や$>130\ \mathrm{keV}$の電子フラックスの増加があった期間では、顕著な\ce{NO}の増加がみられなかった。\par

降り込む粒子の物理量および磁気圏の物理状態と\ce{NO}増加量の関係を明らかにするため、磁気圏と電離層の様子を調べることができるOMNI Data Setを用いた比較を行った。
トロムソ・昭和基地で\ce{NO}の柱密度の増加を4回確認することができたが、その内の2回において、磁気嵐の主相である時期、1回は磁気嵐の回復相かつ高速太陽風の到達した時期であることが分かった。
トロムソにおいては、\ce{NO}の柱密度の増加が確認された2つの時期どちらにおいても、地球磁場の南北成分のゆらぎがあり、サブストームが活発で高速太陽風が吹いていることが確認できた。
トロムソにおいては、2019年2月1日~2019年2月4日における柱密度の増加について、磁気嵐の主相にあたる時期であることが分かった。
昭和基地においては、磁気嵐の発生と対応して\ce{NO}の柱密度が増加した2023年3月23日〜2023年3月24日において地球磁場が南方向を向いており、プロトンの密度も上昇していることが確認できたが、高速太陽風は確認できなかった。
もう1つ柱密度の増加が確認できた2023年3月25日においては、SYM/H(もしくはDst指数)の値をみると磁気嵐は回復相にあたるが、高速太陽風があることが確認できた。
これより、磁気圏から回復する期間であっても高速太陽風が到達していると電子の降り込みがあり、\ce{NO}の増加に影響を与えると考えられる。
磁気嵐の回復相で、磁場の南北成分の振動の中心が南向きのときに高速太陽風が到達する際に、電子の降り込みがあることは先行研究~\cite{miyoshi2013high}でも確認されている。
