\chapter{OMNI Web Data Set}
\label{app:omni}
OMNI Web Data SetはNASAから提供されており、主に磁気圏と電離層における太陽風の影響の調査をサポートすることを目的としたものである~\cite{king2023omni}。
OMNI Web Data Setの中でも本研究で用いたデータは以下の5つである。
\begin{enumerate}
    \item 地球磁場の南北成分(GSM)$[\mathrm{nT}]$
    \item 太陽風の速さ$[\mathrm{m/s}]$
    \item プロトン密度$[\mathrm{n/cc}]$
    \item AE指数$[\mathrm{nT}]$(用いたのはトロムソのみ)
    \item SYM/H$[\mathrm{nT}]$
\end{enumerate} \par
SYM/HはDst指数の1分値に相当するものである~\cite{wdc2009asysym}。
また、AE指数はサブストームに伴う電流の大きさを表すものであり、高緯度オーロラ帯の12か所磁場変動の最大値と最小値の差をとったものである。
% SYM/HについてDst指数との関係を述べる。
% AE指数がない理由を述べる。
