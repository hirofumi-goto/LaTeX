\chapter{OMNI Web Data Set}
\label{app:omni}
OMNI Web Data SetはNASAから提供されており、主に磁気圏と電離層における太陽風の影響の調査をサポートすることを目的としたものである~\cite{king2005solar,king2023omni}。
本研究では、OMNI Web Data Setの中でもHigh Resolution OMNI data set(HRO)を用いた。
HROは1分と5分の時間分解能のデータにより構成されており、はじめは3つの衛星(ACE・Wind・IMP 8)のデータによって構成され($2005-2006$年)、後の2007年にGeotail、2009年にGOESのプロトンのフラックスデータが追加された。
\par
OMNI Web Data Setの中でも本研究で用いたデータは以下の5つである。
\begin{enumerate}
    \item 地球磁場の南北$z$成分$B_z$(GSM)$[\mathrm{nT}]$
    \par
    地球磁場の南北成分を表す。
    GSM(Geocentric Solar Magnetospheric system)とは、座標系の一つである~\cite{russell1971geophysical}。
    地球から太陽までを$x$軸とし、$y$軸は地球の磁気双極子に対して垂直であり、$z$軸は$y-z$平面が双極子軸を含むように定義される。
    なお、$z$軸は北極の磁極の方向を正とする。

    \item 太陽風の速さ$[\mathrm{m/s}]$
    % 太陽風の速さ
    \item プロトン密度$[\mathrm{n/cc}]$
    % プロトン密度
    \item AE指数$[\mathrm{nT}]$
    \par
    AE指数はサブストームに伴う電流の大きさを表すものであり、高緯度オーロラ帯の12か所磁場変動の最大値と最小値の差をとったものである。
    AE指数においてはトロムソの\ce{NO}の柱密度の比較においてのみ用いている。
    昭和基地の解析した期間についてはAE指数は\today 時点で公開されていないため、昭和基地における柱密度との比較ではAE指数は用いていない。

    \item SYM/H$[\ \mathrm{nT}]$
    \par
    SYM/HはDst指数の1分値に相当するものである~\cite{wdc2009asysym}。

\end{enumerate}
