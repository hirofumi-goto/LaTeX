\chapter{Dst指数}
\label{app:dst}
Dst指数のデータは地磁気世界資料センター京都(World Data Center for Geomagnetism, Kyoto)より調べた~\cite{wdc2022creditdst}。
Dst指数とは、地磁気擾乱の大きさを表す指数であり、地磁気擾乱が起きる際に発生する、地球を取り巻く環状の電流(Ring Current)がどの程度地球磁場をどのくらい打ち消すかを表したものである。
Dst指数は、4か所(Hermanus, Kakioka, Honolulu, San Juan)で観測された地磁気の南北成分をもとに1時間値で導出される~\cite{sugiura1986dst}。
実際に地磁気擾乱が発生した場合は、値が急激に減少し、大きな負の値を取る。
また、トロムソの解析結果との比較に用いたDst指数は暫定値(Provisional value)であり、昭和基地の解析結果との比較に用いたDst指数は速報値(Real-time value)である。
暫定値はノイズなどを人目で取り除いたデータから算出されているが、速報値はノイズ除去などの操作を行う前の生データから算出されているため正確ではない値を含んでいることに注意する必要がある~\cite{wdc2022aedst}。
本研究では、地磁気擾乱が起きたかの判断をするための目安として用いた。
