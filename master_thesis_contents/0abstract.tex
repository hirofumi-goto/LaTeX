\begin{abstract}
    太陽活動に伴って極域に降り込む高エネルギー粒子が$\mathrm{NO}_x$(窒素酸化物)やオゾンなどの中層大気中の微量分子に及ぼす影響を観測的に調べるため、我々は2012年から南極・昭和基地(69.00\textdegree S, 39.85\textdegree E)、2016年から北極域のノルウェー・トロムソ(69.35\textdegree N, 19.14\textdegree E)でミリ波分光観測を行っている。
    Isono et al., 2014bでは、ミリ波分光を用いた昭和基地でのNO(一酸化窒素)の観測で、季節変化に伴う長期変動と冬期に高エネルギー粒子の影響による短期変動が確認された。
    しかし、夏はNOの光解離による減少と高エネルギー粒子の影響による増加を切り分けることができなかった。
    そこで本研究では、季節が逆転する北極域を含めた両極域での同時観測による解析の実現を目指す。
    \par
    トロムソでは\ce{NO}の2本の超微細構造線のスペクトルを同時観測していたが、昭和基地で2022年7月から定常観測を開始した多周波数ミリ波分光計のFFT分光計帯域は$2.5\  \mathrm{GHz}$であり、トロムソでの観測で用いた分光計と比べ2.5倍の帯域を持つため、比較的近傍の周波数にある6本のスペクトルの同時観測が可能となった。
    \par
    トロムソについては2018年12月26日から2019年3月10日までの75日間にわたって実施した\ce{NO}のテスト観測、昭和基地については多周波数ミリ波分光計を用いて得られた2023年3月22日から31日までの10日間にわたる\ce{NO}の観測データの中からNO柱密度の導出を行った。
    \ce{NO}が存在する領域の大気温度は一様に$200\ \mathrm{K}$で、\ce{NO}輝線は光学的に薄いと仮定した。
    \par
    トロムソの分光計では積分時間は24時間であったが、\ce{NO}の6本の超微細構造線から導出される柱密度を平均することで、昭和基地における積分時間は12時間と短くした。
    昭和基地での柱密度の誤差の平均はトロムソでの観測と比べて20\ \% 小さくなり、時間分解能を小さくしながら柱密度の誤差を小さくすることができた。
    解析の期間には磁場の擾乱により加速された電子の影響とみられる\ce{NO}の増加が確認できた。
\end{abstract}
