\begin{abstract}
    我々のグループは、太陽活動に伴って極域に降り込む高エネルギー粒子が$\mathrm{NO}_x$(窒素酸化物)やオゾンなどの中層大気中の微量分子に及ぼす影響を観測的に調べるため、2012年から南極・昭和基地(69.00\textdegree S, 39.85\textdegree E)、2016年から北極域のノルウェー・トロムソ(69.35\textdegree N, 19.14\textdegree E)でミリ波分光観測を行っている。
    先行研究(Isono et al., 2014)では、ミリ波分光を用いた昭和基地でのNO(一酸化窒素)のモニタリング観測において、季節変化に伴う長期変動の他、冬期に高エネルギー粒子の影響によると考えられる短期変動が確認された。
    しかし、夏期は高エネルギー粒子の影響だけでなく、太陽光での光解離による減少もあり、これらを切り分けることができなかった。
    そこで本研究では、季節が逆転する北半球の極域にも着目し、両極域での同時観測を実現するために、解析に用いることができるかデータを精査し柱密度の導出を行った。
    導出した\ce{NO}の柱密度の時間変動について、高エネルギー粒子がどのように影響を与えているかを調べた。
    \par
    昭和基地とトロムソでは、ミリ波分光計の仕様は異なっている。
    トロムソでは$1.0\ \mathrm{GHz}$の帯域を持つFFT分光計を使い\ce{NO}の2本の超微細構造線のスペクトルを同時観測しているが、昭和基地の分光計帯域は$2.5\ \mathrm{GHz}$であり、6本のスペクトルの同時観測が可能である。
    \par
    本研究では、これらのスペクトルデータの解析手法を検討した後、トロムソについては2018年12月26日から2019年3月10日までの75日間、昭和基地については2023年3月22日から31日までの10日間にわたる\ce{NO}の観測データの中から\ce{NO}の柱密度の導出を行って、その変動について考察した。
    \par
    トロムソの分光計では積分時間は24時間であったが、\ce{NO}の6本の超微細構造線から導出される柱密度を平均することで、昭和基地における積分時間は12時間と短くした。
    昭和基地での柱密度の誤差の平均はトロムソでの観測と比べて20\% 小さくなり、時間分解能を小さくしながら柱密度の誤差を小さくすることができた。
    解析の期間には磁場の擾乱により加速された電子の影響とみられる\ce{NO}の増加が確認できた。
    \par
\end{abstract}
