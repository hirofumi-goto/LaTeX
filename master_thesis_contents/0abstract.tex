\begin{abstract}
    我々のグループは、太陽活動に伴って極域に降り込む高エネルギー粒子が窒素酸化物($\mathrm{NO}_x$)やオゾンなどの中層大気中の微量分子に及ぼす影響を調べるため、極域でミリ波大気観測を行っている。
    先行研究(Isono et al.2014)では、ミリ波分光を用いた昭和基地での一酸化窒素(\ce{NO})のモニタリング観測において、季節変化に伴う長期変動の他、冬期に高エネルギー粒子の影響によると考えられる短期変動が確認された。
    しかし、夏期は高エネルギー粒子の降り込みだけでなく、太陽光での光解離による影響もあり、これらを切り分けることが難しかった。
    そこで本研究では、季節が逆転する北半球の極域にも着目し、両極域での同時観測を実現するために、実際に取得されたデータを精査して統一的な解析手法の開発を行った。
    さらにこの解析によりNOの柱密度の導出を行い、その時地球に降り込んできた高エネルギー粒子がどのようにNO分子の時間変動に影響を与えているかを調べた。
    \par
    北極域のノルウェー・トロムソ(69.35\textdegree N, 19.14\textdegree E)と南極・昭和基地(69.00\textdegree S,39.85\textdegree E)のミリ波観測データを用いた。昭和基地とトロムソでは、設置されているミリ波分光計の仕様は異なっている。
    トロムソでは$1.0\ \mathrm{GHz}$の帯域を持つFFT分光計を使って\ce{NO}の2本の超微細構造線スペクトルを同時観測しているが、昭和基地の分光計帯域は$2.5\ \mathrm{GHz}$であり、計6本のスペクトルの同時観測が可能である。
    本研究では、これらのスペクトルデータの共通の解析手法を検討した後、トロムソについては2018年12月26日から2019年3月10日までの75日間、昭和基地については2023年3月22日から31日までの10日間にわたる観測データから\ce{NO}の柱密度の導出を行った。
    \par
    解析の結果、トロムソでは24時間の積分時間が必要であったが、昭和基地では\ce{NO}の6本の超微細構造線を全て用いることで、12時間に短縮できることが分かった。
    さらに、昭和基地のデータから求められる柱密度の誤差は、トロムソと比べて20\% 小さくなり、時間分解能を小さくしながら柱密度の誤差を小さくすることに成功した。
    これを地磁気擾乱の指数や衛星観測による降り込み電子のフラックスと比較したところ、磁場の擾乱により加速され、\ce{NO}が存在する高度まで達するエネルギーを持った電子が降り込むことで\ce{NO}が増加した可能性が確認できた。
\end{abstract}
